\documentclass[pdftex]{article}
\usepackage[pdftex]{graphics}
\usepackage{subfigure}
\usepackage{hhline}
\usepackage[usenames,dvipsnames]{color}
\usepackage{colortbl}
\usepackage[screen,pdftex]{mcdlecture}
\newcommand{\bs}{\relax}
\newcommand{\es}{\newpage}
\fboxsep=.01\textwidth \fboxrule=1pt
\newsavebox{\savepar}
\newenvironment{boxit}{\begin{lrbox}{\savepar}
    \begin{minipage}[b]{0.975\textwidth}}
    {\end{minipage}\end{lrbox}\framebox{\usebox{\savepar}}}


%%%%%%%%%%%%%%%%%%%%%%%%%%%%%%%%%%%%%%%%%%%%%%%%%%%%%%%%%%
%% THE FOLLOWING ARE THINGS THAT WE MIGHT CHANGE FROM YEAR TO YEAR OR
%% VENUE TO VENUE
    \lhead{MCMC in Statistical Genetics}
    \lfoot{Dr Eric C. Anderson and Dr Matthew Stephens}
%	\lfoot{Dr Eric C. Anderson and Dr John Novembre}
    \rfoot{UW - Summer Institute, July 2015}
%	 \rfoot{Edinburgh - European Institute, June 2012}
%\rfoot{Brazil - Summer Institute, February 2014}

% on this one, be sure to update the venue and the module number
%\newcommand{\coursetitlepage}{European Institute in Statistical Genetics
\newcommand{\coursetitlepage}{Summer Institute in Statistical Genetics
%\newcommand{\coursetitlepage}{Brazilian Edition of the Summer \\Institute in Statistical Genetics

Module 18:

MCMC for Genetics}

%% Then update the schedule.  Note that I have broken that
%% out into a separate file like: schedule_table_edinburgh2012.tex
%% which is input in Overview.tex

%% Then be sure to change any time-sensitive events in the 
%% probability discussion in Matthew's first lecture.

%% And also update "structure_fun" link to my wiki to the right
%% year and venue.
%%%%%%%%%%%%%%%%%%%%%%%%%%%%%%%%%%%%%%%%%%%%%%%%%%%%%%%%%%


\begin{document}

\DeclareGraphicsExtensions{.jpg,.pdf,.png}%



%% Eric added a few things:
% some commands that Eric made for making a title while starting
% a new lecture and for making titles of new slides.
\newcommand{\newlecture}[1]{\newpage\begin{center}\section*{#1}\end{center}}
\newcommand{\newslide}[1]{\newpage\subsection*{#1: \hfil}}
 \newcommand{\Exp}{\Bbb{E}}
 \newcommand{\Var}{{\mathrm{Var}}}
 %% Some pretty etc.'s, etc...
\newcommand{\cf}{{\em cf.}}
\newcommand{\eg}{{\em e.g.},}
\newcommand{\ie}{{\em i.e.},}
\newcommand{\etal}{{\em et al.}\ }
\newcommand{\etc}{{\em etc.}}

%% some handy things for making bold math
\def\bm#1{\mathpalette\bmstyle{#1}}
\def\bmstyle#1#2{\mbox{\boldmath$#1#2$}}
\newcommand{\thh}{^\mathrm{th}}
\newcommand{\bpi}{{\pi}}
\newcommand{\mP}{\mathbf{P}}
\rhead{References - \thepage}

\section*{References}


Anderson, E. C. and Thompson, E. A. (2002). A model-based method for identifying species hybrids using multilocus genetic data, {\em Genetics}, {\bf 160}, 1217--1229.



Brooks, S. and Gelman, A.  (1998). General methods for monitoring convergence
of iterative simulations.  {\em Journal of Computational and Graphical
Statistics.}


Clark A.G. (1990). Inference of haplotypes from PCR-amplified
samples of diploid populations. {\em Molecular Biology and Evolution}, {\bf 7}, 111--122.


Falush, D., Stephens, M. and Pritchard, J. K. (2003)
Inference of population structure using multilocus genotype data: linked loci and correlated allele frequencies,
  {\em Genetics},
{\bf 164},
1567-1587.






Feller, W. (1957). {\em An Introduction to Probability Theory and Its Applications}, 2nd Edition. New York, John Wiley \& Sons.

Geman, S. and Geman, D. (1984). Stochastic relaxation, Gibbs distributions, and the Bayesian restoration of images." {\em IEEE Transactions on Pattern Analysis and Machine Intelligence}, {\bf 6}, 721--741.

Gelman, A. and Meng, X.-L. (1998). Simulating normalizing constants: from importance sampling to bridge sampling to path sampling.  {\em Statistical Science}, {\bf 13}, 163--185.

Gelman, A. and Rubin, D.  (1992).  Inference from iterative simulation using
multiple sequences,  {\em Statistical Science} {\bf 7}, 457--511.

Gelman, A., Carlin, J.B., Stern, H.B. and Rubin D.B.  (2004).  {\em Bayesian
Data Analysis}, (2nd  Ed).  New York:  Chapman \& Hall.

Geweke, J.  (1992). Evaluating the accuracy of sampling-based approaches to
the calculation of posterior moments.  In J. M. Bernardo, A. F. M. Smith, A.
P. Dawid and J. O. Berger (eds.), {\em Bayesian Statistics 4}, Oxford
University Press.

Geyer, C. J. (1991). Markov chain Monte Carlo maximum likelihood. {\em Computing Science and Statistics: Proceedings of the 23rd Symposium on the   Interface,}  Interface Foundation, Fairfax Station, 156--163.

Geyer, C. J. and Thompson, E. A. (1992). Constrained Monte Carlo maximum likelihood for dependent data (with discussion). {\em Journal of the Royal Statistical Society, Series B}, {\bf 54}, 657--699.

Gilks, W. R. and Wild, P. (1992). Adaptive rejection sampling for Gibbs
sampling. {\em Applied Statistics}, {\bf 41}, 337--48.

Green, P. J. (1995). Reversible jump Markov chain Monte Carlo computation and Bayesian model   determination,  {\em Biometrika}, {\bf 82}, 711--732.

Griffiths, R. C. and Tavare, S. (1994). Simulating probability distributions in the coalescent. {\em Theoretical Population Biology}, {\bf 46}, 131--159.

Guo, S. W. and Thompson, E. A. (1992).  Performing the exact test of Hardy-Weinberg proportion for multiple alleles. {\em Biometrics}, {\bf 48}, 361--372.

Hastings, W. K. (1970). Monte Carlo sampling methods using Markov chains and their applications. {\em Biometrika}, {\bf 57}, 97--109.

Heidelberger, P. and Welch, P.D. (1983).  Simulation run length control in
the presence of an initial transient. {\em Operations Research} {\bf 31},
1109--1144.

Kass, R.E. and Raftery, A.E.  (1995). Bayes factors. {\em Journal of the
American Statistical Association}, {\bf 90}, 773--795.

Levene, H. (1949). On a matching problem arising in genetics. {\em The Annals of Mathematical Statistics}, {\bf 20}, 91--94.

Metropolis, N., Rosenbluth, A. W., Rosenbluth, M. N., Teller, A. H. and Teller, E. (1953). Equations of state calculations by fast computing machines. {\em Journal of Chemical Physics} {\bf 21}, 1087--1092.

Metropolis, N. and Ulam, S.  (1949).  The Monte Carlo method.  {\em Journal of the American Statistical Association}, {\bf 44}, 335--341.



Pritchard, J. K., Stephens, M. and Donnelly, P. (2000).
 Inference of population structure using multilocus genotype data,
 {\em Genetics},
  {\bf 155}, 945--959.






Raftery, A. E. and Lewis, S.M. (1992).  How Many Iterations in the Gibbs
Sampler?  In J. M. Bernardo, A. F. M. Smith, A. P. Dawid and J. O. Berger
(eds.), {\em Bayesian Statistics 4}, Oxford University Press.


Ripley, B. D. (1987). {\em Stochastic Simulation},  New York: Wiley \& Sons.



Schwarz, G.  (1978).  Estimating the dimension of a model. {\em Annals of
Statistics}, 6(2), 461-464.

Spiegelhalter, D.J., Best, N.G., Carlin, B.P. and van der Linde, A.  (2002).
Bayesian measures of model complexity and fit (with discussion).  {\em
Journal of the Royal Statistical Society}, Series B, {\bf 64},4,583--640.

Stephens, M. and Donnelly, P. (2000). Inference in molecular population genetics (with Discussion), {\em Journal of the Royal Statistical Society, Series B}, {\bf 62}, 605--655.


Wilson, G A and Rannala, B  (2003).
  Bayesian inference of recent migration rates using multilocus genotypes,
  {\em Genetics},
  {\bf 163}, 1177--1191.




Wright, S. and McPhee, H. C. (1925) An approximate method of calculating coefficients of inbreeding and relationship from livestock pedigrees. {\em Journal of Agricultural Research}, {\bf 31}, 377--383


Yu, B. and Mykland, P. A.  (1998).  Looking at Markov samplers through cusum
path plots: a simple diagnostic idea.  {\em Statistics and Computing}, {\bf
8} 275--286.

\end{document}

