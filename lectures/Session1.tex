\documentclass[pdftex]{article}
\usepackage[pdftex]{graphics}
\usepackage{subfigure}
\usepackage{hhline}
\usepackage[usenames,dvipsnames]{color}
\usepackage{colortbl}
\usepackage[screen,pdftex]{mcdlecture}
\newcommand{\bs}{\relax}
\newcommand{\es}{\newpage}
\fboxsep=.01\textwidth \fboxrule=1pt
\newsavebox{\savepar}
\newenvironment{boxit}{\begin{lrbox}{\savepar}
    \begin{minipage}[b]{0.975\textwidth}}
    {\end{minipage}\end{lrbox}\framebox{\usebox{\savepar}}}


%%%%%%%%%%%%%%%%%%%%%%%%%%%%%%%%%%%%%%%%%%%%%%%%%%%%%%%%%%
%% THE FOLLOWING ARE THINGS THAT WE MIGHT CHANGE FROM YEAR TO YEAR OR
%% VENUE TO VENUE
    \lhead{MCMC in Statistical Genetics}
%   \lfoot{Dr Eric C. Anderson and Dr Matthew Stephens}
%    \lfoot{Dr Eric C. Anderson and Dr Matthew Robinson}
    \lfoot{Dr Matthew Stephens and Dr Matthew Robinson}
%	\lfoot{Dr Eric C. Anderson and Dr John Novembre}
    \rfoot{UW - Summer Institute, July 2019}
%	 \rfoot{Edinburgh - European Institute, June 2012}
%\rfoot{Brazil - Summer Institute, February 2014}

% on this one, be sure to update the venue and the module number
%\newcommand{\coursetitlepage}{European Institute in Statistical Genetics
\newcommand{\coursetitlepage}{Summer Institute in Statistical Genetics
%\newcommand{\coursetitlepage}{Brazilian Edition of the Summer \\Institute in Statistical Genetics

Module 18:

MCMC for Genetics}

%% Then update the schedule.  Note that I have broken that
%% out into a separate file like: schedule_table_edinburgh2012.tex
%% which is input in Overview.tex

%% Then be sure to change any time-sensitive events in the 
%% probability discussion in Matthew's first lecture.

%% And also update "structure_fun" link to my wiki to the right
%% year and venue.
%%%%%%%%%%%%%%%%%%%%%%%%%%%%%%%%%%%%%%%%%%%%%%%%%%%%%%%%%%


\begin{document}

\DeclareGraphicsExtensions{.jpg,.pdf,.png}%



%% Eric added a few things:
% some commands that Eric made for making a title while starting
% a new lecture and for making titles of new slides.
\newcommand{\newlecture}[1]{\newpage\begin{center}\section*{#1}\end{center}}
\newcommand{\newslide}[1]{\newpage\subsection*{#1: \hfil}}
 \newcommand{\Exp}{\Bbb{E}}
 \newcommand{\Var}{{\mathrm{Var}}}
 %% Some pretty etc.'s, etc...
\newcommand{\cf}{{\em cf.}}
\newcommand{\eg}{{\em e.g.},}
\newcommand{\ie}{{\em i.e.},}
\newcommand{\etal}{{\em et al.}\ }
\newcommand{\etc}{{\em etc.}}

%% some handy things for making bold math
\def\bm#1{\mathpalette\bmstyle{#1}}
\def\bmstyle#1#2{\mbox{\boldmath$#1#2$}}
\newcommand{\thh}{^\mathrm{th}}
\newcommand{\bpi}{{\pi}}
\newcommand{\mP}{\mathbf{P}}
\rhead{Session 1 - \thepage}

\section*{\hfil Personal Probability}

If I take a shuffled pack of cards, what is the probability
that the top card is the Ace of Spades? 

Are you sure?

What would change your mind?

\es\bs

\section*{\hfil Personal Probability}

If I toss a coin 10 times, and see 0 heads, what is your estimate
for the long-run probability of a head?

What if I toss a coin once, and see 0 heads?

\es\bs

\section*{\hfil Personal Probability}

In Classical statistics, probability is treated as a way of stating
limiting long-run frequencies of independent events occuring.

In Bayesian statistics, probability is treated as a way ({\it the} way)
of expressing uncertainty about unknown quantities.

This view is more flexible, since it allows one to use
probability in treating uncertain events of many different kinds.
For example, what is the probability 
that it will rain tomorrow? That the world will have a safe, effective,
vaccine for SARS-CoV2 in 2021. 
That Monrovia is the capital of Liberia?

Uncertainty is, intrinsically, personal. Individuals may differ in their uncertainty, and therefore
in their personal probabilities.   For this reason personal probabilities are also
referred to as ``subjective probabilities." 

\es\bs

\section*{Personal Probability: Underlying Theory}

The theory underlying personal probability is too vast for us to
do justice here.

Fortunately you don't need to know it to apply or understand much of Bayesian statistics
(although it helps).

Here's a one-line summary: ``In the face of uncertainty, a logical person should
act as if they are assigning probabilities to uncertain events, losses to outcomes,
and act so as to minimize their expected loss." 

And if I'm allowed another line: ``Probabilities of uncertain events should be updated in the light
of new information, using Bayes Theorem".

For more info (non-math): Lindley, ``Understanding Uncertainty".



\es\bs


\section*{\hfil Classical Inference\hfil}

Probability model $p\left( {\left. y \right|\theta } \right)$
 for data $y = \left( {y_1 , \ldots ,y_n } \right)$.

$\theta $ a vector of parameters {\em assumed fixed but unknown}.

 Statistical inference is concerned with estimating the value of $\theta $
 from the data $y$.

Statistical procedure such as maximum likelihood estimation (MLE) provides
estimate $\hat \theta$  of $\theta$  as a function of $y$.

MLE is the value of $\theta $ which maximises $p\left( {\left. y
\right|\theta } \right)$, regarded as a function of $\theta $
 (the likelihood function).
    \es \bs

{\em Example: Estimating population allele frequency}  

Suppose $n$ diploid individuals sampled from a population, typed
at a locus with two alleles (A,a). $y$ = number of A alleles observed. Want
to estimate $\theta$, frequency of $A$ allele.

Model the alleles as being independent draws from $\Pr(A) = \theta$ (Hardy Weinberg Equilibrium).

$$L(\theta) = p\left( {\left. y \right|\theta } \right) = \left( {\begin{array}{*{20}c}
   2n  \\
   y  \\
\end{array}} \right)\theta ^y \left( {1 - \theta } \right)^{2n - y}
,\qquad   y = 0, 1,\ldots, 2n$$

\begin{equation*}
\hat \theta  = y/2n
\end{equation*}

Uncertainty (eg confidence intervals) estimated from sampling distribution of $\hat \theta$.
    \es\bs

\section*{\hfil Bayesian Inference}

In Bayesian inference, unknown parameters $\theta $
 are regarded as random variables rather than fixed but unknown quantities.

Prior knowledge about $\theta $
 expressed by a ``prior" probability distribution $p\left( \theta  \right)$.

Information in observed data $y$ expressed via the Likelihood function
$L(\theta) = p\left( {\left. y \right|\theta } \right)$.

Two sources of information are combined using Bayes Theorem (or ``Bayes Rule"):
\begin{equation*}
p\left( {\left.
\theta  \right|y} \right) = \frac{{p\left( {\left. y \right|\theta }
\right)p\left( \theta \right)}}{{p\left( y \right)}} 
\end{equation*}

$p\left( {\left. \theta  \right|y} \right)$ called the ``posterior distribution" of
$\theta $.

\es\bs

\section*{\hfil Bayes Rule: Derivation}

The joint probability of $y$ and $\theta $
 can be expressed as either
$$p\left( {y,\theta } \right) = p\left( {\left. y \right|\theta } \right)p\left( \theta
\right)$$
    or
$$
p\left( {\theta ,y} \right) = p\left( {\left. \theta  \right|y}
\right)p\left( y \right).
$$

Equating these gives Bayes rule.

    \es\bs

\es\bs
\section*{Bayes Rule: Convenient Form}

\begin{equation*}
p\left( {\left.
\theta  \right|y} \right) = \frac{{p\left( {\left. y \right|\theta }
\right)p\left( \theta \right)}}{{p\left( y \right)}} 
\end{equation*}

$$\mbox{posterior} \propto \mbox{likelihood} \times \mbox{prior}.$$
 \es\bs

{\em Example: Estimating allele frequency} 

 $n$ diploid individuals sampled from a population, typed
at a locus with two alleles (A,a). $y$ = number of A alleles. Want
to estimate $\theta$, frequency of $A$ allele.

$$\mbox{posterior} \propto \mbox{likelihood} \times \mbox{prior}.$$

Likelihood: as before (Hardy--Weinberg Equilibrium).

Prior: For parameters like an allele frequency, constrained to lie in $[0,1)$, a convenient prior is
the beta distribution: $\theta \sim \mbox{Beta}\left( {\alpha ,\beta }\right)$, with density
$$p\left( \theta ; \alpha, \beta \right) = \frac{{\Gamma
\left( {\alpha  + \beta } \right)}}{{\Gamma \left( \alpha  \right)\Gamma
\left( \beta  \right)}}\theta ^{\alpha  - 1} \left( {1 - \theta }
\right)^{\beta  - 1} ,\quad  0 < \theta  < 1; \alpha>0; \beta>0$$

\section*{Beta Distribution: summary}

$$p(\theta; \alpha, \beta) \propto \theta^{\alpha-1} (1-\theta)^{\beta-1}$$

 $E\left( \theta \right) = \frac{\alpha }{{\alpha  + \beta }}.$

 ${\mathop{\rm var}} \left( \theta  \right) = \frac{{\alpha \beta }}{{\left( {\alpha  + \beta } \right)^2 \left( {\alpha  + \beta  + 1} \right)}}.$

Large values of $\alpha,\beta$ give small variance: $\theta$ becomes peaked about its expectation.

Small values of $\alpha, \beta$ ($<1$) give peaks at 0,1.

$\alpha=\beta=1$ gives uniform distribution on [0,1).

\es\bs
%\centerline{\resizebox{0.75\textwidth}{!}{
	\begin{center}
		\includegraphics[width=.7\textwidth]{figures/betaplot}
	\end{center}
    \es\bs

{\em Continuing with our example:}

$$\mbox{posterior} \propto \mbox{likelihood} \times \mbox{prior}.$$

$$p\left( {\left. y \right|\theta }
\right) \propto \theta ^y \left( {1 - \theta } \right)^{2n - y} $$

$$p\left( \theta  \right) \propto \theta ^{\alpha  - 1} \left( {1 - \theta }
\right)^{\beta  - 1}.$$

The posterior is  
$$p\left( {\left. \theta  \right|y} \right) \propto \theta ^{\alpha  + y - 1}
\left( {1 - \theta } \right)^{\beta  + 2n - y - 1},$$
 the beta distribution with parameters $\alpha  + y$
 and $\beta  + 2n - y$.
 
 $$\theta | y \sim \text{Be}(\alpha+y, \beta + 2n -y).$$

\es\bs

 \subsection*{Summarizing posteriors: Point Estimates}

The posterior distribution of $\theta $ encapsulates our
beliefs about $\theta $ after seeing the data (and in light of our prior beliefs). 

It can be helpful to produce point estimates of $\theta$. 

The Bayesian approach to this involves specifying a ``loss" function
$L(\theta,\hat{\theta})$ that says how much you ``lose" by guessing
$\hat{\theta}$ if the truth is $\theta$. Then choose the value of $\hat{\theta}$ that
minimizes the (posterior) expected loss:
$$E[L(\theta,\hat{\theta})] = \sum_t L(t,\hat{\theta}) p(\theta= t | y).$$
[Note, for continuous parameters the sum is replaced by an integral].

The estimate that minimizes this expected loss is called the {\it Bayes Estimate}.
\es\bs

{\color{section0}\em Example: Posterior mean}

If the loss function is ``squared loss": $L(\theta,\hat{\theta}) = (\theta-\hat{\theta})^2$
then the Bayes Estimate is the {\em posterior mean}:

$$E( \theta |y ) = \sum_t t p(\theta=t  |y).$$

In our allele frequency example, the posterior mean is
$$E\left( {\left. \theta  \right|y} \right) = \frac{{\alpha  + y}}{{\alpha  + \beta  + 2n}}$$

Compare with the prior mean $\frac{\alpha }{{\alpha  + \beta }}$
 and the mle $\frac{y}{2n}.$
    \es\bs

{\color{section0}\em Example: Posterior mode}

If the loss function is ``0-1 loss": 
$$L(\theta,\hat{\theta}) = 0 \text{ if $\theta = \hat{\theta}$; 1 otherwise}$$
then the Bayes Estimate is the {\em posterior mode} (the ``most
likely'' posterior value).

The mode of the $\mbox{Beta}\left( {\alpha ,\beta } \right)$
 distribution is $\frac{{\alpha  - 1}}{{\alpha  + \beta  - 2}}$.
    
{\em Example}

In our example the posterior mode is $\frac{{\alpha  + y - 1}}{{\alpha  +
\beta  + 2n - 2}}$.

So if $\alpha=\beta=1$, posterior mode is $\frac{y}{2n}$.

So, with a uniform prior, the posterior mode = MLE. (Follows
from posterior $\propto$ likelihood $\times$ prior.)
    \es\bs
    
 \subsection*{A note about nuisance parameters}
   
  Consider an inference problem where you are interested
   in $\theta$, and there is a nuisance parameter $\eta$. (e.g. estimating a normal
   mean, with an unknown variance).
   
   The classical maximum likelihood approach maximizes the likelihood over both $\theta$ and $\eta$.
    
    The Bayesian approach is to compute the posterior $p(\theta,\eta  | y)$, and then to compute the
    posterior for $\theta$ by integrating out $\eta$:
    $$p(\theta | y) = \int p(\theta,\eta  | y) d\eta.$$ 
    
    You then use this ``marginal distribution" to get a point estimate of $\theta$ (e.g. posterior mean or mode.) Note that here a uniform prior on $(\theta,\eta)$ does not necessarily give posterior mode($\theta$) = mle.


\subsection*{Summarizing Posterior: interval estimates}

A $95\%$ ``credible interval" for $\theta$ is an interval $I$ with the property
that $\Pr(\theta \in I | y) = 0.95$.

Such intervals are most easily formed by taking the $2.5\%$ and $97.5\%$ percentiles of the posterior
density. This yields a ``symmetric" or ``central" CI.

An alternative (usually considered superior, but harder to do)
is to the {\em highest posterior density} (HPD) region:
\begin{itemize}
\item[] The set of values of $\theta $
 that contains $95\%$ of the posterior probability, and such that the density within the region is always higher than the density outside.
\end{itemize}
    \es\bs
    
{\color{section0}\em Bayesian vs frequentist intervals}

 Strict interpretation of classical (frequentist) confidence interval is awkward:
\begin{itemize}
\item[] For {\em fixed} (unknown) parameter $\theta$, 95\% confidence limits are
functions of the data $y$, say $a(y)$ and $b(y)$, satisfying
$$P\left( {\left. {a\left( y \right) < \theta  < b\left( y \right)} \right|\theta } \right) = 0.95.$$

Here $y$ is the random variable and we have to imagine infinite repetitions
of the experiment to give meaning to the probability statement.
\end{itemize}
For the corresponding Bayesian credible interval, $\theta$ satisfies
$$P\left( {\left. {a\left( y
\right) < \theta  < b\left( y \right)} \right|y} \right) = 0.95$$ where
$\theta $ is now the random variable.

\es\bs

\section*{\hfil Conjugate Priors}

In our example, both the prior and posterior for $\theta$ were Beta
distributions.

This made for a particularly simple analysis.

This property of a prior has a special name: the Beta distribution
is the ``conjugate" prior for this problem (binomial sampling).

Conjugate priors are often used, because of the simplicity they
introduce. However, it is not necessary to restrict oneself to
a conjugate prior.

\es\bs

\subsection*{Considerations for choice of Prior}

Prior for $\theta $
 should capture our knowledge before observing data.  Possible sources:
\begin{itemize}
    \item background scientific knowledge;
    \item previous studies;
    \item expert judgements (see e.g. O'Hagan, 1998, and other papers in the same
issue).
\end{itemize}

\es\bs

When ``testing" a large number of hypotheses, it can be helpful to think about how
many total positive findings one might expect when trying to assess the prior probability that
any one is true.

{\it Example:} in ongoing genome-wide association studies, a million genetic variants (SNPs) across the genome may be tested for association with a complex disease (eg diabetes). 

Perhaps tens or hundreds of them will actually be causally linked to the disease.

Therefore a reasonable prior might be in the range $10/ 10^6$ to $100/10^6$.

\es\bs

{\it Example:} in a candidate gene study one might select, say, 10 genes to study in detail,
to assess whether genetic variation in these genes are associated with disease. 

Presumably the study would not have been done had it not been considered at least
reasonably likely that one of the genes would be associated with the disease.

Based on this, a prior probability of 1/10 for each gene containing genetic variation associated with the disease, might be reasonable.

    \es\bs

\subsection*{Example 2: Assignment Problem}

Consider the problem of estimating the origin of an individual, based
on genetic data. 

One bi-allelic locus with alleles $A, a$.

Allele frequencies of $A$ in two populations are $f_1$ and $f_2$.

What is the probability that an individual with genotype $AA$ came
from population 1 vs population 2?

That is, what is $\Pr(\text{Pop 1} | AA)$?

\es\bs
We need a prior probability that the individual came from population 1 vs 2.

Assume the two are equally likely {\it a priori}, so prior probability is 0.5.

Then apply Bayes Theorem:

$$\Pr(\text{Pop 1} | AA) \propto \Pr(AA | \text{Pop 1}) \Pr(\text{Pop 1}) = f_1^2 0.5$$

$$\Pr(\text{Pop 2} | AA) \propto \Pr(AA | \text{Pop 2}) \Pr(\text{Pop 2}) = f_2^2 0.5$$

These must sum to 1, so this gives us the constant of proportionality:

$$\Pr(\text{Pop 1} | AA) = 0.5 f_1^2 / (0.5 f_1^2 + 0.5 f_2^2)$$
$$\Pr(\text{Pop 2} | AA) = 0.5 f_2^2 / (0.5 f_1^2 + 0.5 f_2^2)$$

\es\bs

\subsection*{Example: Assignment Problem with estimated allele frequencies}

Assume now that $f_1$ and $f_2$ are unknown, but
are to be estimated from 10 ``reference" individuals sampled from each population.

Suppose that in 10 individuals, we see 1 copy of A in population 1, and 0 copies of A
in population 2. 

What is the estimated probability that an AA individual came from population 1?

Naive Frequentist Analysis: answer = ?

Naive Bayes Analysis: answer = ?
\es\bs

\section*{\hfil Bayesian Inference: summary}

Bayesian inference proceeds by
\begin{enumerate}
    \item specifying what is known about $\theta$ before obtaining the data as
the prior $p\left( \theta \right)$;
    \item collecting the data $y$ and
specifying its distribution $p\left( {\left. y \right|\theta } \right)$ (or
the likelihood function);
    \item calculating the posterior $p\left( {\left.
\theta \right|y} \right)$ from Bayes rule.
\end{enumerate}

%
%\subsection*{Example: multiallelic loci}

%Here we suppose we observe $n$
% independent trials each of which has $p$
% possible outcomes with associated probabilities
%\[\theta  = (\theta _1 , \ldots ,\theta _p ).\]
%Letting $y = (y_1 , \ldots ,y_p )$
% denote the number of times each outcome is observed, we have
%\begin{eqnarray*}
%  y|\theta  &  \sim  & {\rm Multinomial}(n,\theta ) \cr
%  p(y|\theta ) &  =  & \frac{{(\sum\nolimits_i  \,y_i )!}}{{\prod\nolimits_i  \,y_i !}}\prod\limits_i  \,\mathop {\theta _i }\nolimits^{y_i } ;\quad \quad \sum\limits_i  \,y_i  = n,\sum\limits_i  \,\theta _i  = 1
%\end{eqnarray*}
%\es\bs
%Placing a Dirichlet prior on $\theta $
% will give a posterior distribution for $\theta $
% which is also Dirichlet.

%If a-priori we assume
%\[
%\theta  \sim {\rm Dirichlet}(\alpha _{01} , \ldots ,\alpha _{0p}
%);\quad \quad \alpha _{0i}  > 0
%\]
%then
%\[
%p(\theta ) \propto \prod\limits_i  \,\theta _i^{\alpha _{0i}  - 1}
%\]
%and hence
%\[
%p(\theta |y) \propto \prod\limits_i  \,\theta _i^{y_i  + \alpha
%_{0i}  - 1}
%\]
%so that if we let $\alpha _{1i}  = \alpha _{0i}  + y_i $
%\[
%(\theta |y) \sim {\rm Dirichlet}(\alpha _{11} , \ldots ,\alpha
%_{1p} ).
%\]
%\es
%\bs
%From the properties of the Dirichlet it is easy to show that the
%marginal posterior distribution of each $\theta _i $
% will be
%\[
%{\rm Beta}\left( {\alpha _{1i} ,\sum\limits_{j \ne i}  \,(\alpha
%_{1j} )} \right)
%\]
%Hence,
%\begin{eqnarray*}
%  {\rm E}[\theta _i |y] &  =  & \mu _i  = \frac{{\alpha _{1i} }}{{\sum\nolimits_{j = 1}^p  \,\alpha _{1j} }}, \cr
%  {\rm Var}[\theta _i |y] &  =  & \frac{{\alpha _{1i} (\sum\nolimits_{j \ne i}  \,\alpha _{1j} )}}{{(\sum\nolimits_{j = 1}^p  \,\alpha _{1j} )^2 (1 + \sum\nolimits_{j = 1}^p  \,\alpha _{1j} )}}.
%\end{eqnarray*}
%\es\bs
%We can also show the correlation between $\theta _i $ and $\theta
%_j $ is given by
%\[
%{\rm Corr}(\theta _i ,\theta _j ) =  - \sqrt {\left( {\frac{{\mu
%_i }}{{1 - \mu _i }}} \right)\left( {\frac{{\mu _j }}{{1 - \mu _j
%}}} \right)} .
%\]

%If any of these correlations are large looking at the marginal
%distributions of $\theta _i $
% and $\theta _j $
% separately would be very misleading.
%\es\bs
%\subsection*{Summary}

%In multiparameter problems our aim is still to obtain the
%posterior distribution of the (vector) parameter using Bayes' rule
%to combine the prior distribution with the likelihood. This leads
%to the need to perform lots of integrations which cannot be done
%analytically except in some special cases. In general we will need
%to use numerical integration methods or avoid the integrations
%entirely by using simulation methods (see later).

%Characterising the information contained in a multidimensional
%distribution is difficult, so we usually resort to low-dimensional
%(i.e. 1-D) marginal distributions as summaries although this can
%be misleading.
\end{document}
