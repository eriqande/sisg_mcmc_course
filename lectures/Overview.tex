\documentclass[pdftex]{article}
\usepackage[pdftex]{graphics}
\usepackage{subfigure}
\usepackage{hhline}
\usepackage[usenames,dvipsnames]{color}
\usepackage{colortbl}
\usepackage[screen,pdftex]{mcdlecture}
\newcommand{\bs}{\relax}
\newcommand{\es}{\newpage}
\fboxsep=.01\textwidth \fboxrule=1pt
\newsavebox{\savepar}
\newenvironment{boxit}{\begin{lrbox}{\savepar}
    \begin{minipage}[b]{0.975\textwidth}}
    {\end{minipage}\end{lrbox}\framebox{\usebox{\savepar}}}


%%%%%%%%%%%%%%%%%%%%%%%%%%%%%%%%%%%%%%%%%%%%%%%%%%%%%%%%%%
%% THE FOLLOWING ARE THINGS THAT WE MIGHT CHANGE FROM YEAR TO YEAR OR
%% VENUE TO VENUE
    \lhead{MCMC in Statistical Genetics}
    \lfoot{Dr Eric C. Anderson and Dr Matthew Stephens}
%	\lfoot{Dr Eric C. Anderson and Dr John Novembre}
    \rfoot{UW - Summer Institute, July 2015}
%	 \rfoot{Edinburgh - European Institute, June 2012}
%\rfoot{Brazil - Summer Institute, February 2014}

% on this one, be sure to update the venue and the module number
%\newcommand{\coursetitlepage}{European Institute in Statistical Genetics
\newcommand{\coursetitlepage}{Summer Institute in Statistical Genetics
%\newcommand{\coursetitlepage}{Brazilian Edition of the Summer \\Institute in Statistical Genetics

Module 18:

MCMC for Genetics}

%% Then update the schedule.  Note that I have broken that
%% out into a separate file like: schedule_table_edinburgh2012.tex
%% which is input in Overview.tex

%% Then be sure to change any time-sensitive events in the 
%% probability discussion in Matthew's first lecture.

%% And also update "structure_fun" link to my wiki to the right
%% year and venue.
%%%%%%%%%%%%%%%%%%%%%%%%%%%%%%%%%%%%%%%%%%%%%%%%%%%%%%%%%%


\begin{document}

\DeclareGraphicsExtensions{.jpg,.pdf,.png}%



%% Eric added a few things:
% some commands that Eric made for making a title while starting
% a new lecture and for making titles of new slides.
\newcommand{\newlecture}[1]{\newpage\begin{center}\section*{#1}\end{center}}
\newcommand{\newslide}[1]{\newpage\subsection*{#1: \hfil}}
 \newcommand{\Exp}{\Bbb{E}}
 \newcommand{\Var}{{\mathrm{Var}}}
 %% Some pretty etc.'s, etc...
\newcommand{\cf}{{\em cf.}}
\newcommand{\eg}{{\em e.g.},}
\newcommand{\ie}{{\em i.e.},}
\newcommand{\etal}{{\em et al.}\ }
\newcommand{\etc}{{\em etc.}}

%% some handy things for making bold math
\def\bm#1{\mathpalette\bmstyle{#1}}
\def\bmstyle#1#2{\mbox{\boldmath$#1#2$}}
\newcommand{\thh}{^\mathrm{th}}
\newcommand{\bpi}{{\pi}}
\newcommand{\mP}{\mathbf{P}}
\rhead{Overview - \thepage}

\vspace*{1in}
\begin{center}\color{section0}\bf\Large
\coursetitlepage
\end{center}
\es\bs

\begin{center}\color{section0} Presenters\end{center}
\enlargethispage*{1000pt}

{\small
{\color{section0}Dr. Eric C. Anderson\\
NOAA - Southwest Fisheries Science Center, Santa Cruz, CA\\
Research Associate, Department of Applied Math \& Statistics, UCSC}

Eric develops methods for the analysis of genetic data from populations of salmon and other organisms.  He has developed approaches using MCMC and other Monte Carlo methods for estimating $N_e$, classifying species hybrids, the inference of pedigrees, and analysis of massively parallel sequencing data.

{\small
{\color{section0}Dr. Matthew Stephens\\
Professor, Department of Human Genetics and\\
Department of Statistics, The University of Chicago}	

Matthew Stephens develops methods for the analysis of genetic data in humans and other species. He has, among other applications, developed MCMC methods to estimate population structure, rates of recombination, missing genotypes, and haplotype phase from genotypes on unrelated individuals.

% {\color{section0}Dr. Matthew Robinson \\
% Assistant Professor, Complex Trait Genetics Group\\
% Department of Computational Biology \\
% University of Lausanne, Switzerland}
%   
% Matthew Robinson began working on evolutionary genetics in wild populations and then changed fields to human 
% medical genetics.  His research areas include 
% prediction of disease risk in personalised medicine, the genetics of aging,
% genotype-environment interactions, the role of selection in shaping human
% population differentiation, and assortative mating of human populations.

}

\es\bs
\begin{center}
{\color{section0}\bf\Large Course Web Materials}
\end{center}
\vspace*{.25in}
\enlargethispage*{1000pt}

We have a page on GitHub with links to a variety of materials relevant to the course.
You can get there here:

\url{http://eriqande.github.io/sisg_mcmc_course/}

\es\bs
\begin{center}
{\color{section0}\bf\Large Schedule of Sessions}
\vspace*{.25in}
\enlargethispage*{1000pt}

%% Here we put in the schedule
%\begin{tabular}{ll}
\multicolumn{2}{c}{\bf Monday}\\
Session 1  &   Intro, Basic Theory, and the Bayesian Perspective \\
Session 2  &   Monte Carlo and Markov Chains\\
Session 3  &   Simple MCMC Examples Using R\\
Session 4  &   More Complex MCMC Examples\\
\\
\multicolumn{2}{c}{\bf Tuesday}\\
Session 5  &   Case Study I:   Inference of Population Structure\\
Session 6  &   Practical on {\em structure}\\
Session 7  &   Case Study II:   Haplotype Inference\\
Session 8  &   Practical on PHASE\\
\\
\multicolumn{2}{c}{\bf Wednesday}\\
Session 9   &   Bayes Factors and Model Selection\\
Session 10  &   Importance Sampling, MCMC Likelihood, \\
 & Metropolis-coupled MCMC\\
\end{tabular}

\begin{tabular}{ll}
\multicolumn{2}{c}{\bf Monday}\\
8:30-10:00am   &Session 1: Intro, Basic Theory, and\\
                          &the Bayesian Perspective \\
10:00-10:30am  &BREAK\\
10:30-12noon   &Session 2: Monte Carlo and Markov Chains\\
12:00-1:30pm   &LUNCH\\
1:30-3:00pm      &Session 3: Simple MCMC Examples Using R\\
3:00-3:30pm    &BREAK\\
3:30-5:00pm    &Session 4: More Complex MCMC Examples\\
& \\
\multicolumn{2}{c}{\bf Tuesday}\\

8:30-10:00am   &Session 5: Case Study I: Inference of Population \\
				&Structure\\
10:00-10:30am  &BREAK\\
10:30-12noon   &Session 6: Practical on {\em structure}\\
12:00-1:30pm   &LUNCH\\
\end{tabular}



\es\bs
\begin{tabular}{ll}
\multicolumn{2}{c}{\bf Tuesday (cont'd)}\\
1:30-3:00pm      &Session 7: Case Study II: Haplotype Inference\\
3:00-3:30pm    &BREAK\\
3:30-5:00pm    &Session 8: Practical on PHASE\\
&\\
\multicolumn{2}{c}{\bf Wednesday}\\
8:30-10:00am &Session 9: Bayes Factors and Model Selection\\
10:00-10:30am  &BREAK\\
10:30-12noon   &Session 10: Importance Sampling, MCMC Likelihood, \\ & Pedigrees, 
				Metropolis-coupled MCMC\\
\end{tabular}

%\begin{tabular}{ll}
\multicolumn{2}{c}{\bf Wednesday}\\

12:00-1:30pm   &LUNCH\\
1:30-3:00pm      &Session 1: Intro, Basic Theory, and\\
                          &the Bayesian Perspective \\
3:00-3:30pm    &BREAK\\
3:30-5:00pm    &Session 2: Monte Carlo and Markov Chains\\
& \\
\multicolumn{2}{c}{\bf Thursday}\\

8:30-10:00am   &  Session 3: Simple MCMC Examples Using R\\
10:00-10:30am  &BREAK\\
10:30-12noon   & Session 4: More Complex MCMC Examples\\
12:00-1:30pm   &LUNCH\\
1:30-3:00pm      &  Session 5: Case Study I: Inference of Population \\
				&Structure\\
3:00-3:30pm    &BREAK\\
3:30-5:00pm    &  Session 6: Practical on {\em structure}\\
\end{tabular}



\es\bs
\begin{tabular}{ll}
\multicolumn{2}{c}{\bf Friday}\\
8:30-10:00am &Session 7: Case Study II: Haplotype Inference\\
10:00-10:30am  &BREAK\\
10:30-12noon   &Session 8: Practical on PHASE\\
12:00-1:30pm   &LUNCH\\
1:30-3:00pm      &Session 9: Bayes Factors and Model Selection\\
3:00-3:30pm    &BREAK\\
3:30-5:00pm    &Session 10: Importance Sampling, MCMC Likelihood, \\ & Pedigrees, 
				Metropolis-coupled MCMC\\
\end{tabular}







\end{center}
\end{document}

