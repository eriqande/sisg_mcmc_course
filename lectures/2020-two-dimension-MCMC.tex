% put all Mike's macros in and begin document
\documentclass[pdftex]{article}
\usepackage[pdftex]{graphics}
\usepackage{subfigure}
\usepackage{hhline}
\usepackage[usenames,dvipsnames]{color}
\usepackage{colortbl}
\usepackage[screen,pdftex]{mcdlecture}
\newcommand{\bs}{\relax}
\newcommand{\es}{\newpage}
\fboxsep=.01\textwidth \fboxrule=1pt
\newsavebox{\savepar}
\newenvironment{boxit}{\begin{lrbox}{\savepar}
    \begin{minipage}[b]{0.975\textwidth}}
    {\end{minipage}\end{lrbox}\framebox{\usebox{\savepar}}}


%%%%%%%%%%%%%%%%%%%%%%%%%%%%%%%%%%%%%%%%%%%%%%%%%%%%%%%%%%
%% THE FOLLOWING ARE THINGS THAT WE MIGHT CHANGE FROM YEAR TO YEAR OR
%% VENUE TO VENUE
    \lhead{MCMC in Statistical Genetics}
    \lfoot{Dr Eric C. Anderson and Dr Matthew Stephens}
%	\lfoot{Dr Eric C. Anderson and Dr John Novembre}
    \rfoot{UW - Summer Institute, July 2015}
%	 \rfoot{Edinburgh - European Institute, June 2012}
%\rfoot{Brazil - Summer Institute, February 2014}

% on this one, be sure to update the venue and the module number
%\newcommand{\coursetitlepage}{European Institute in Statistical Genetics
\newcommand{\coursetitlepage}{Summer Institute in Statistical Genetics
%\newcommand{\coursetitlepage}{Brazilian Edition of the Summer \\Institute in Statistical Genetics

Module 18:

MCMC for Genetics}

%% Then update the schedule.  Note that I have broken that
%% out into a separate file like: schedule_table_edinburgh2012.tex
%% which is input in Overview.tex

%% Then be sure to change any time-sensitive events in the 
%% probability discussion in Matthew's first lecture.

%% And also update "structure_fun" link to my wiki to the right
%% year and venue.
%%%%%%%%%%%%%%%%%%%%%%%%%%%%%%%%%%%%%%%%%%%%%%%%%%%%%%%%%%


\begin{document}

\DeclareGraphicsExtensions{.jpg,.pdf,.png}%



%% Eric added a few things:
% some commands that Eric made for making a title while starting
% a new lecture and for making titles of new slides.
\newcommand{\newlecture}[1]{\newpage\begin{center}\section*{#1}\end{center}}
\newcommand{\newslide}[1]{\newpage\subsection*{#1: \hfil}}
 \newcommand{\Exp}{\Bbb{E}}
 \newcommand{\Var}{{\mathrm{Var}}}
 %% Some pretty etc.'s, etc...
\newcommand{\cf}{{\em cf.}}
\newcommand{\eg}{{\em e.g.},}
\newcommand{\ie}{{\em i.e.},}
\newcommand{\etal}{{\em et al.}\ }
\newcommand{\etc}{{\em etc.}}

%% some handy things for making bold math
\def\bm#1{\mathpalette\bmstyle{#1}}
\def\bmstyle#1#2{\mbox{\boldmath$#1#2$}}
\newcommand{\thh}{^\mathrm{th}}
\newcommand{\bpi}{{\pi}}
\newcommand{\mP}{\mathbf{P}}

\rhead{Session 7 -- MCMC in more dimensions -- \thepage}




% each \include command puts a new file in.

% to make typesetting faster while working on a single
% section, but still have the appropriate page numbering
% and references, use the \includeonly command



\newlecture{The Metropolis-Hastings algorithm \\
in more than one dimension}
Goals of this lecture:
\begin{itemize}
\item Introduce a simple model for ``inbreeding''
\item Consider MCMC in more than a single dimension
\item Describe component-wise Metropolis-Hastings sampling 
\end{itemize}

\newslide{Genotype Frequencies and Inbreeding}
Starting with the last exercise of Session~3: one locus with two alleles, $A$ and $a$, at frequencies $p$ and $1-p$, respectively, and ``inbreeding coefficient'' $f$.

Probabilities of the three genotypes are:
\begin{itemize}
\item $P(AA) = fp + (1-f)p^2$
\item $P(Aa~\mathrm{or}~aA) = (1-f)2p(1-p)$
\item $P(aa) = f(1-p) + (1-f)(1-p)^2$
\end{itemize}

Since ``inbreeding'' is used to describe a lot of (related) things, let's briefly review what this model is saying\ldots




%%%% INBREEDING DIAGRAMS
\newslide{Inbreeding model}
\enlargethispage*{1000pt}
\begin{minipage}[c]{.45\textwidth}
\begin{center}
Not-inbred with probability \\
$1-f$

\includegraphics[width=.8\textwidth]{illus/SimpleNotInbreeding.pdf}
\end{center}
\begin{itemize}
\small
\item $P(AA) = p^2$
\item $P(Aa~\mathrm{or}~aA) = 2p(1-p)$
\item $P(aa) = (1-p)^2$
\end{itemize}

\end{minipage}
\hfill
\begin{minipage}[c]{.45\textwidth}
\begin{center}
Inbred with probability \\
$f$

\includegraphics[width=.8\textwidth]{illus/SimpleInbreeding.pdf}
\end{center}
\begin{itemize}
\small
\item $P(AA) = p $
\item $P(Aa~\mathrm{or}~aA) = 0$
\item $P(aa) = (1-p)$
\end{itemize}
\end{minipage}
\hrule
{\small
\begin{eqnarray*}
P(n_{AA},n_{Aa},n_{aa}|p,f) &=& C\times [fp + (1-f)p^2]^{n_{AA}} \times [(1-f)2p(1-p)]^{n_{Aa}} \\
 &\times & [ f(1-p) + (1-f)(1-p)^2]^{n_{aa}}
\end{eqnarray*}
}




\newslide{What Our Data Would Look Like}
\begin{itemize}
\item We have a sample of $n$ individuals total.
\begin{itemize}
\item $n_{AA}$ are homozygous for the $A$ allele
\item $n_{Aa}$ are heterorozygous
\item $n_{aa}$ are homozygous for the $a$ allele
\end{itemize}
\item Clearly, $n=n_{AA} + n_{Aa} + n_{aa}$
\end{itemize}

A concrete example we will use throughout this lecture is $n=50$ with:
\[
n_{AA}=30~~~~~~~~~~~~~n_{Aa}=10~~~~~~~~~~~~~n_{aa}=10
\]
which are roughly the expected values if $p=.7$ and $f=.5$.




%%BAYESIAN SPECIFICATION FOR ESTIMATING f and p
\newslide{Bayesian Model for Estimating $f$ and $p$}
To go about estimating $f$ and $p$ in a Bayesian fashion, we must fully specify the model.  This means that we need 
\begin{itemize}
\item Priors for $f$ and $p$.  We will choose uniform priors $f\sim\mathrm{Beta}(1,1)$ and $p\sim\mathrm{Beta}(1,1)$
\item The data themselves.  These are $n_{AA}$, $n_{Aa}$, and $n_{aa}$.
\item The likelihood: $P(n_{AA},n_{Aa},n_{aa}|p,f)$, which is given on the previous page
\end{itemize}
The posterior distribution is then prior $\times$ likelihood, divided by the (``nasty'') normalizing constant
\[
P(p,f|n_{AA},n_{Aa},n_{aa}) = \frac{P(f)P(p)P(n_{AA},n_{Aa},n_{aa}|p,f)}{\int_{f,p}P(f)P(p)P(n_{AA},n_{Aa},n_{aa}|p,f)dfdp}
\]


\newpage
Computing the normalizing constant is difficult, but with MCMC, {\em we don't have to!}

Recall, to perform MCMC it suffices to know the target distribution up to a constant of proportionality. 

Our target distribution is 
\[
P(p,f|n_{AA},n_{Aa},n_{aa}) \propto P(f)P(p)P(n_{AA},n_{Aa},n_{aa}|p,f)
\]
So long as $0< f< 1$ and $0<p<1$.

Otherwise it is 0.


\newslide{Two-Dimensional M-H Sampler}
We can compute the target distribution (up to a constant) easily for any $f$ and $p$.   So, to simulate from this posterior distribution we can just implement a Metropolis-Hastings sampler.  

Following the examples of Session~3, for $p$ we will choose a normal proposal distribution centered on the current value with standard deviation of $s_p$:
\[
	q(p^*|p) \equiv \mathrm{Normal}(p,s_p)
\]
And, we will use the same for $f$:
\[
	q(f^*|f) \equiv \mathrm{Normal}(f,s_f)
\]

Applying these proposal distributions in sequence gives us a simple way to simulate proposed values, $(p^*,f^*)$, from the current values $(p,f)$.

\newpage

A ``sweep'' of our MCMC algorithm would look like:
\begin{enumerate}
\item Propose a new value, $(p^*,f^*)$ for $(p,f)$
\begin{itemize}
\item propose $p^*$ from $\mathrm{Normal}(p,s_p)$
\item propose $f^*$ from $\mathrm{Normal}(f,s_f)$
\end{itemize}
\item Accept or reject the proposed value $(p^*,f^*)$ with probability $R$ which is the minimum of 1 and the Hasting's Ratio:
\end{enumerate}
\[
\small
R=\min\biggr\{1,
\frac{q(p|p^*)q(f|f^*)}{q(p^*|p)q(f^*|f) } \times 
\frac{P(p^*)P(f^*)P(n_{AA},n_{Aa},n_{aa}|p^*,f^*)}{P(p)P(f)P(n_{AA},n_{Aa},n_{aa}|p,f)}
\biggl\}
\]
If you accept the proposed value, set the current value to the proposed value.  Otherwise leave the current values unchanged.

\fbox{Computer Demo: {\tt inbred\_p -n 30 10 10} {\small (starts on ``Jointly'' (j))}}


\newslide{Component-wise Metropolis Hastings Sampler  }
\begin{itemize}
\item In any MCMC implementation, the proposal distribution {\em need not propose changes to {\bf every} variable/parameter  in the model}.
\item In fact, there are few ``real-world'' problems requiring MCMC in which you would use a single proposal distribution in which changes were proposed to all the variables in the model.
\item {\bf Important Concept:}
\begin{itemize}
\item Any proposal distribution, regardless of how many or how few variables it proposes changes to, is valid, so long as the proposal is accepted or rejected in a way that satisfies detailed balance w.r.t.\ the target distribution.
\item These different flavors of the ``propose-reject/accept'' step may be combined in series in whatever manner is desired, so long as they produce an irreducible, aperiodic chain\footnote{Nonetheless, some ways are better than others and will lead to a better-mixing chain}. 
\end{itemize}
\end{itemize}


\newslide{Simple Component-wise M-H Sampler for $p$ and $f$}
Simplest scenario has a sweep as follows:
\begin{enumerate}
\item Do an update for \textcolor{Violet}{$p$}:
\begin{enumerate}
\item propose $\textcolor{Violet}{p^*}$ from $\mathrm{Normal}(p,s_p)$
\item Accept or reject the proposed value $\textcolor{Violet}{p^*}$ with probability $\min\{1,\alpha\}$, where: 
\[
\alpha = \frac{\textcolor{Violet}{q(p|p^*)}}{\textcolor{Violet}{q(p^*|p)} } \times
\frac{\textcolor{Violet}{P(p^*)}P(f)\textcolor{Violet}{P(n_{AA},n_{Aa},n_{aa}|p^*,f)}}{\textcolor{Violet}{P(p)}P(f)\textcolor{Violet}{P(n_{AA},n_{Aa},n_{aa}|p,f)}}
\]
\end{enumerate}
\item Do an update for \textcolor{Orange}{$f$}:
\begin{enumerate}
\item propose $\textcolor{Orange}{f^*}$ from $\mathrm{Normal}(f,s_f)$
\item Accept or reject the proposed value $\textcolor{Orange}{f^*}$ with probability $\min\{1,\alpha\}$, where: 
\[
\alpha = \frac{
	\textcolor{Orange}{q(f|f^*)}}  {\textcolor{Orange}{q(f^*|f)} } \times
\frac{
	P(p)\textcolor{Orange}{P(f^*)}\textcolor{Orange}{P(n_{AA},n_{Aa},n_{aa}|p,f^*)}}
{P(p)\textcolor{Orange}{P(f)}\textcolor{Orange}{P(n_{AA},n_{Aa},n_{aa}|p,f)}}
\]
\end{enumerate}
\end{enumerate}



\newslide{Simple Component-wise M-H Sampler for $p$ and $f$, cont'd}
Some very important points:
\begin{itemize}
\item The proposals are each a little simpler (though just slightly\ldots) than jointly proposing changes to $(p,f)$
\item Neither step~1 nor step~2 of the sweep creates an irreducible chain (obviously, if you never update $p$, for example, your chain could never reach every possible value of $p$).
\item However, taken together, steps~1 and~2 create an irreducible chain.
\end{itemize}
\fbox{Computer Demo: {\tt inbred\_p} {\small (Component-wise using (c) from info window)}}


\newslide{Simple Component-wise M-H Sampler for $p$ and $f$, cont'd}
Two more important points
\begin{itemize}
\item The form of the Hastings ratio is a little simpler when we have proposed changing just a subset of the variables.
\item However, in this case the target density remains just as complex, because it does not factorize into a separate part for $f$ and a part for $p$.
\item Since we are changing just a small part of the model at a time, it seems like we could spend some more energy on making each separate proposal distribution more ``intelligent.''
\end{itemize}

The final two points above get us to thinking about Gibbs sampling, which we will return to after a brief discussion of latent variables\ldots



\newslide{Wrap-Up}
Main Points:
\begin{itemize}
\item In problems where it is useful, MCMC almost always proceeds by proposing changes to a small subset of the variables.
\item There may be many different proposal types.
\begin{itemize}
\item Each proposal type {\em must} satisfy detailed balance.
\item Each proposal type need not make an irreducible chain, BUT
\item All proposals taken together should form an irreducible chain.
\end{itemize}
\end{itemize}





\end{document}